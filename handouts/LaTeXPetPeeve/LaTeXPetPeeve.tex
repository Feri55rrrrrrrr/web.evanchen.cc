\documentclass[10pt]{scrartcl}
\usepackage{evan}
\usepackage{booktabs}
\renewcommand{\thepage}{}
\addtolength{\textheight}{9em}
\addtolength{\voffset}{-6em}

\begin{document}
\title{How to annoy Evan with \LaTeX}
\subtitle{A list of pet peeves}
\maketitle

\bgroup\footnotesize
It is assumed you are using \texttt{amsmath} and \texttt{amssymb} packages,
which you likely are already if you are trying to type math.
See also \url{https://web.evanchen.cc/latex-style-guide.html}.
\egroup

\begin{center}
\begin{tabular}[h]{llllc}
  \toprule
  \multicolumn{2}{c}{Instead of\dots} & \multicolumn{2}{c}{Annoy Evan by using \dots} & Notes \\
  \midrule
  \verb#``quotes''# & ``quotes'' & \verb#"quotes"# & "quotes" & \\
  \verb#$\sin(x)$# & $\sin(x)$ & \verb#$sin(x)$# & $sin(x)$ & \eqref{item:operator} \\
  \verb#$1,\dots,n$# & $1,\dots,n$ & \verb#$1,...,n$# & $1,...,n$ & \eqref{item:dots} \\
  \verb#$1,\dots,n$# & $1,\dots,n$ & \verb#$1,\cdots,n$# & $1,\cdots,n$ & \eqref{item:dots} \\
  \verb#$a$, $b$, and $c$# & $a$, $b$, and $c$ & \verb#$a,b,$ and $c$# & $a,b,$ and $c$ & \eqref{item:comma} \\
  \verb#$p \mid n$# & $p \mid n$ & \verb#$p | n$# & $p | n$ & \eqref{item:mid} \\
  \verb#$\ell \parallel m$# & $\ell \parallel m$ & \verb#$\ell || m$# & $\ell || m$ & \\
  \verb#$a \pmod n$# & $a \pmod n$ & \verb#$a (\text{mod }n)$# & $a (\text{mod }n)$ & \eqref{item:mod} \\
  \verb#$2 \cdot 3 = 6$# & $2 \cdot 3 = 6$ & \verb#$2 * 3 = 6$# & $2 * 3 = 6$ & \\
  \verb#$2 \times 3 = 6$# & $2 \times 3 = 6$ & \verb#$2$x$3 = 6$# & $2$x$3 = 6$ & \\
  \verb#$\left< x,y \right>$# & $\left< x,y\right>$ & \verb#$<x,y>$# & $<x,y>$ & \eqref{item:lr} \\
  \verb#\[ 1+1=2 \]# & See \eqref{item:display} & \verb#$$1+1=2$$# & See \eqref{item:display} & \eqref{item:display} \\
  \bottomrule
\end{tabular}
\end{center}
\subsection*{Notes}
\begin{enumerate}
  \ii \label{item:operator}
  This also applies to $\cos$, $\tan$, $\gcd$, $\min$, $\max$, $\deg$,
  $\log$, $\ln$, $\exp$, $\inf$, $\sup$, \dots.
  (For custom operators, say $\lcm(a,b)$, write \verb#$\operatorname{lcm}(a,b)$#.
  Or put \verb#\DeclareMathOperator{\lcm}{lcm}# in the preamble to define \verb#\lcm#.)

  \ii \label{item:dots}
  Generally, you should almost always use \verb#\dots#, even outside math mode.
  The two dots commands, \verb#\ldots# ($\ldots$) and \verb#\cdots# ($\cdots$) put the dots
  in different places.
  Generally, you want the former for lists and text, the latter between operators.
  The smarter \verb#\dots# will auto-detect which case you are in.

  \ii \label{item:comma}
  The spacing right before the variable $b$ is affected.

  \ii \label{item:mid}
  Also in set notation, e.g.
  $\left\{ x \mid f(x) > 0 \right\}$ is \verb#$\left\{ x \mid f(x) > 0 \right\}$#.

  \ii \label{item:mod}
  \verb#$a \mod n$# gives ``$a \mod n$'',
  \verb#$a \bmod n$# gives ``$a \bmod n$''.

  \ii \label{item:lr}
  \verb#\left# and \verb#\right# are also used for resizing
  \verb+()+, \verb+[]+, \verb+\{\}+ to match heights of tall inputs.
  Compare \verb+\[ f\left( \frac12 \right) \]+ and \verb+\[ f( \frac12 ) \]+:
  \[ f\left( \half \right) \qquad \text{vs.} \qquad f(\half) . \]

  \ii \label{item:display}
  \verb+$$...$$+ is a \TeX{} primitive, not officially supported by \LaTeX.
  It ``usually'' works, but there are
  \href{https://tex.stackexchange.com/q/503/76888}{occasional mysterious breakages}
  (whereas \verb#\[ ... \]# always works).
  For example, the \verb#\qedhere# command will break:
  \begin{proof}[Example proof with double dollar signs]
    Follows by $$1 + 1 = 2. \qedhere$$
  \end{proof}
  \begin{proof}[Example proof with correct syntax]
    Follows by \[ 1 + 1 = 2. \qedhere \]
  \end{proof}
\end{enumerate}

% display

\end{document}
