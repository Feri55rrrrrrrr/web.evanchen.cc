\documentclass[11pt]{scrartcl}
\usepackage{evan}

\begin{asydef}
  import graph;
  real[] default_ticks = {-4,-3,-2,-1,0,1,2,3,4};
  void cplane(real theta=-45, real[] T = default_ticks, real xmin = -5, real xmax = 5, real S = 3) {
    graph.xaxis("Re", xmin, xmax, graph.Ticks(format="%", Ticks=T, Size=S), Arrows);
    graph.yaxis("Im", xmin, xmax, graph.Ticks(format="%", Ticks=T, Size=S), Arrows);
    dot("$0$", (0,0), dir(theta));
  }
\end{asydef}

\begin{document}
\title{Bashing Geometry with Complex Numbers}
\date{29 August 2015}
\maketitle

\begin{abstract}
  This is a (quick) English translation of the complex numbers note
  I wrote for Taiwan IMO 2014 training.
  Incidentally I was also working on an airplane.
\end{abstract}

\section{The Complex Plane}
Let $\CC$ and $\RR$ denote the set of complex and real numbers, respectively.

Each $z \in \CC$ can be expressed as
\[ z = a+bi = r \left( \cos \theta + i \sin \theta \right) = re^{i \theta} \]
where $a,b,r,\theta \in \RR$ and $0 \le \theta < 2\pi$.
We write $\left\lvert z \right\rvert = r = \sqrt{a^2+b^2}$ and $\arg z = \theta$.

More importantly, each $z$ is associated with a conjugate $\ol z = a-bi$.
It satisfies the properties
\begin{align*}
  \ol {w \pm z} &= \ol w \pm \ol z \\
  \ol {w \cdot z} &= \ol w \cdot \ol z  \\
  \ol {w / z} &= \ol w / \ol z \\
  \left\lvert z \right\rvert^2 &= z \cdot \ol z
\end{align*}
Note that $z \in \RR \iff z = \ol z$ and $z \in i \RR \iff z + \ol z = 0$.

\begin{figure}[ht]
  \centering
  \begin{asy}
    size(7cm);
    cplane(135);
    pair z = (3,4);
    pair p = (z.x,0);
    pair x = (z.x,-z.y);
    dot("$z = 3+4i$", z, dir(45));
    dot("$\overline z = 3-4i$", x, dir(-45));
    dot("$-1-2i$", (-1,-2), dir(225));
    draw(x--(0,0)--z);
    dot("$\left\lvert z \right\rvert = 5$", 0.7*z, rotate(90)*dir(z));
    draw(z--x, dotted);
    markangle(L="$\theta$", p, (0,0), z, n=1);
    markrightangle(z, p, (0,0));
    // MarkAngle("\theta", p, (0,0), z, 1);
    // draw(rightanglemark(z,p,(0,0)));
  \end{asy}
  \caption{Points $z = 3+4i$ and $-1-2i$; $\ol z = 3-4i$ is the conjugate.}
  \label{fig:explain_complex_plane}
\end{figure}

We represent every point in the plane by a complex number.
In particular, we'll use a capital letter (like $Z$) to denote the point associated to a complex number (like $z$).

Complex numbers add in the same way as vectors.
The multiplication is more interesting: for each $z_1, z_2 \in \CC$ we have
\[
  \left\lvert z_1z_2 \right\rvert = \left\lvert z_1 \right\rvert \left\lvert z_2 \right\rvert
  \text{ and }
  \arg z_1z_2 = \arg z_1 + \arg z_2.
\]
This multiplication lets us capture a geometric structure.
For example, for any points $Z$ and $W$ we can express rotation of $Z$ at $W$ by $90\dg$ as
\[ z \mapsto i(z-w) + w. \]

\begin{figure}[ht]
  \centering
  \begin{asy}
    size(11cm);
    cplane();
    pair z = (3,4);
    pair iz = rotate(90)*z;
    dot("$z = 3+4i$", z, dir(90));
    dot("$iz = -4+3i$", iz, dir(90));
    draw(z--(0,0)--iz,Arrows(SimpleHead,9));
    markrightangle(z, (0,0), iz);
    // draw(rightanglemark(z,(0,0),iz));
    picture pic = CC();
    cplane();
    pair z = (-1,-2);
    pair w = (-2,-4);
    pair f = rotate(90)*(z-w)+w;
    pair zw = z-w;
    pair izw = rotate(90)*(z-w);
    dot("$z$", z, dir(0));
    dot("$w$", w, dir(-90));
    dot("$i(z-w)+w$", f, dir(100));
    dot("$z-w$", zw, dir(45));
    dot("$i(z-w)$", izw, dir(135));
    draw(f--w--z, Arrows(SimpleHead,9));
    draw(izw--(0,0)--(z-w), Arrows(SimpleHead,9));
    add(shift((-11,0))*pic);
  \end{asy}
  \caption{$z \mapsto i(z-w) + w$.}
  \label{fig:complex_rotate}
\end{figure}


\section{Elementary Propositions}
First, some fundamental formulas:

\begin{proposition}
  Let $A$, $B$, $C$, $D$ be pairwise distinct points.
  Then $\ol{AB} \perp \ol{CD}$ if and only if $\frac{d-c}{b-a} \in i \RR$; i.e.\
  \[ \frac{d-c}{b-a} + \ol{\left( \frac{d-c}{b-a} \right)} = 0. \]
  \label{lem:complex_perp}
\end{proposition}
\begin{proof}
  It's equivalent to
  $\frac{d-c}{b-a} \in i \RR
  \iff \arg \left( \frac{d-c}{b-a} \right) \equiv \pm 90\dg \iff \ol{AB} \perp \ol{CD}$.
\end{proof}

\begin{figure}[ht]
  \centering
  \begin{asy}
    size(12cm);
    cplane();
    pair a = (2,2); dot("$a$", a, dir(120));
    pair b = (3,4); dot("$b$", b, dir(120));
    pair c = (-1,1); dot("$c$", c, dir(90));
    pair d = (-4,2.5); dot("$d$", d, dir(90));
    draw(a--b); draw(c--d);
    add(shift(-11)*CC());
    cplane();
    pair x = b-a; dot("$b-a$", x, dir(90));
    pair y = d-c; dot("$d-c$", y, dir(90));
    draw(x--(0,0)--y);
  \end{asy}
  \caption{$\ol{AB} \perp \ol{CD} \iff \frac{d-c}{b-a} \in i \RR$.}
\end{figure}

\begin{proposition}
  Let $A$, $B$, $C$ be pairwise distinct points.
  Then $A$, $B$, $C$ are collinear if and only if $\frac{c-a}{c-b} \in \RR$; i.e.
  \[ \frac{c-a}{c-b} = \ol{\left( \frac{c-a}{c-b} \right)}. \]
  \label{prop:complex_collin}
\end{proposition}
\begin{proof}
  Similar to the previous one.
\end{proof}

\begin{proposition}
  Let $A$, $B$, $C$, $D$ be pairwise distinct points.
  Then $A$, $B$, $C$, $D$ are concyclic if and only if
  \[ \frac{c-a}{c-b} : \frac{d-a}{d-b} \in \RR. \]
\end{proposition}
\begin{proof}
  It's not hard to see that
  $\arg \left( \frac{c-a}{c-b} \right) = \angle ACB$
  and $\arg \left( \frac{d-a}{d-b} \right) = \angle ADB$.
  (Here angles are directed).
\end{proof}

Now, let's state a more commonly used formula.

\begin{lemma}[Reflection About a Segment]
  Let $W$ be the reflection of $Z$ across $\ol{AB}$. Then
  \[ w = \frac{(a-b) \ol z + \ol{a}b - a\ol{b}}{\ol a - \ol b}. \]
  \label{lem:complex_reflect}
\end{lemma}
Of course, it then follows that the foot from $Z$ to $\ol{AB}$ is exactly $\half (w+z)$.

\begin{figure}[ht]
  \centering
  \begin{asy}
    size(4cm);
    real[] T = {0,2,4};
    cplane(T=T, xmin=-1.9,xmax=5);
    dot("$1$", (2,0), dir(-90));
    pair a = (1,1); dot("$a$", a, dir(-90));
    pair b = (4,3); dot("$b$", b, dir(10));
    pair z = (2,3); dot("$z$", z, dir(90));
    pair w = 2*foot(z,a,b)-z; dot("$w$", w, dir(-45));
    draw(a--b);
    draw(z--w, dotted);
  \end{asy}
  \quad
  \begin{asy}
    size(4cm);
    real[] T = {0,2,4};
    cplane(T=T, xmin=-1.9,xmax=5);
    dot("$1$", (2,0), dir(-90));
    pair a = (0,0);
    pair b = (3,2); dot("$b-a$", b, dir(10));
    pair z = (1,2); dot("$z-a$", z, dir(90));
    pair w = 2*foot(z,a,b)-z; dot("$w-a$", w, dir(45));
    draw(a--b);
    draw(z--w, dotted);
  \end{asy}
  \quad
  \begin{asy}
    size(4cm);
    real[] T = {0,1,2};
    cplane(T=T, xmin=-0.95,xmax=2.50);
    pair a = (0,0);
    pair x = (3,2);
    pair b = (3,2)/x; dot("$1$", b, dir(-90));
    pair z = (1,2)/x; dot("$\frac{z-a}{b-a}$", z, dir(70));
    pair w = 2*foot(z,a,b)-z; dot("$\frac{w-a}{b-a}$", w, dir(-70));
    draw(a--b);
    draw(z--w, dotted);
    draw((0,0)--(1,0), black+1);
  \end{asy}
  \caption{The reflection of $Z$ across $\ol{AB}$.}
  \label{fig:proof_reflect}
\end{figure}

\begin{proof}
  According to \Cref{fig:proof_reflect} we obtain
  \[ \frac{w-a}{b-a} = \ol{\left( \frac{z-a}{b-a} \right)} = \frac{\ol z - \ol a}{\ol b - \ol a}. \]
  From this we derive $w = \frac{(a-b)\ol z + \ol a b - a \ol b}{\ol a-  \ol b}$.
\end{proof}

Here are two more formulas.

\begin{theorem}
  [Complex Shoelace] Let $A$, $B$, $C$ be points.
  Then $\triangle ABC$ has signed area
  \[
    \frac{i}{4}
    \left\lvert
    \begin{array}{ccc}
      a & \ol a & 1 \\
      b & \ol b & 1 \\
      c & \ol c & 1 \\
    \end{array}
    \right\rvert.
  \]
  \textbf{In particular, $A$, $B$, $C$ are collinear if and only if this determinant vanishes.}
  \label{thm:complex_shoelace}
\end{theorem}
\begin{proof}
  Cartesian coordinates.
\end{proof}
Often, \Cref{thm:complex_shoelace} is easier to use than \Cref{prop:complex_collin}.

Actually, we can even write down the formula for an arbitrary intersection of lines.
\begin{proposition}
  \label{prop:complex_intersect}
  Let $A$, $B$, $C$, $D$ be points. Then lines $AB$ and $CD$ intersect at
  \[ \frac{ (\bar a b - a \bar b )(c-d) - (a-b)(\bar c d - c \bar d) }{(\bar a - \bar b )(c-d) - (a-b)(\bar c - \bar d)}. \]
\end{proposition}
But unless $d=0$ or $a$, $b$, $c$, $d$ are on the unit circle, this formula is often too messy to use.


\section{The Unit Circle, and Triangle Centers}
\label{sec:unitcircle}
On the complex plane, the \textbf{unit circle} is of critical importance.
Indeed if $\left\lvert z \right\rvert = 1$ we have \[ \ol z = \frac 1z. \]
Using the above, we can derive the following lemmas.

\begin{lemma}
  If $ \left\lvert a \right\rvert = \left\lvert b \right\rvert = 1$ and $z \in \CC$,
  then the reflection of $Z$ across $\ol{AB}$ is
  $a+b-ab \ol z$,
  and the foot from $Z$ to $\ol{AB}$ is
  \[ \half \left( z+a+b -ab \ol z \right). \]
\end{lemma}
\begin{lemma}
  If $A$, $B$, $C$, $D$ lie on the unit circle then
  the intersection of $\ol{AB}$ and $\ol{CD}$ is given by
  \[ \frac{ab(c+d)-cd(a+b)}{ab-cd}. \]
\end{lemma}

These are much easier to work with than the corresponding formulas in general.
We can also obtain the triangle centers immediately:
\begin{theorem}
  Let $ABC$ be a triangle center, and assume that the circumcircle of
  $ABC$ coincides with the unit circle of the complex plane.
  Then the circumcenter, centroid, and orthocenter of $ABC$
  are given by $0$, $\frac 13 (a+b+c)$, $a+b+c$, respectively.
\end{theorem}
Observe that the Euler line follows from this.

\begin{proof}
  The results for the circumcenter and centroid are immediate.
  Let $h = a+b+c$.
  By symmetry it suffices to prove $\ol{AH} \perp \ol{BC}$.
  We may set
  \[ z  = \frac{h-a}{b-c} = \frac{b+c}{b-c}. \]
  Then
  \[ \ol z = \ol{\left( \frac{b+c}{b-c} \right)}
    = \frac{\ol b + \ol c}{\ol b - \ol c}
    = \frac{\frac 1b + \frac 1c}{\frac 1b - \frac 1c}
    = \frac{c+b}{c-b} = -z \]
  so $z \in i \RR$ as desired.
\end{proof}

We can actually even get the formula for the incenter.
\begin{theorem}
  Let triangle $ABC$ have incenter $I$ and circumcircle $\Gamma$.
  Lines $AI$, $BI$, $CI$ meet $\Gamma$ again at $D$, $E$, $F$.
  If $\Gamma$ is the unit circle of the complex plane
  then there exists $x,y,z \in \CC$ satisfying
  \[ a=x^2, b=y^2, c=z^2 \text{ and } d=-yz, e=-zx, f=-xy. \]
  Note that $\left\lvert x \right\rvert = \left\lvert y \right\rvert = \left\lvert z \right\rvert = 1$.
  Moreover, the incenter $I$ is given by $-(xy+yz+zx)$.
\end{theorem}
\begin{proof}
  Show that $I$ is the orthocenter of $\triangle DEF$.
\end{proof}

\section{Some Other Lemmas}
\begin{lemma}
  Let $A$, $B$ be on the unit circle
  and select $P$ so that $\ol{PA}$, $\ol{PB}$ are tangents.
  Then \[ p = \frac{2}{\ol a + \ol b} = \frac{2ab}{a+b}. \]
\end{lemma}
\begin{proof}
  Let $M$ be the midpoint of $\ol{AB}$ and set $O = 0$.
  One can show $OM \cdot OP = 1$ and that $O$, $M$, $P$ are collinear;
  the result follows from this.
\end{proof}
\begin{figure}[ht]
  \centering
  \begin{asy}
    size(4cm);
    draw(unitcircle);
    pair A = dir(110); dot("$a$", A, dir(110));
    pair B = dir(30); dot("$b$", B, dir(30));
    pair P = 2/conj(A+B);
    dot("$\frac{2ab}{a+b}$", P, dir(P));
    dot(origin);
    draw(A--P--B);
    draw(A--B, dashed);
    draw(origin--P, dashed);
    dot( (A+B)/2 );
  \end{asy}
  \caption{Two tangents. $p = \frac{2}{\ol a + \ol b}$.}
\end{figure}

\begin{lemma}
  For any $x$, $y$, $z$, the circumcenter of $\triangle XYZ$ is given by
  \[ \left\lvert
    \begin{array}{ccc}
      x & x\bar x & 1 \\
      y & y\bar y & 1 \\
      z & z\bar z & 1
    \end{array}
  \right\rvert
  \div
  \left\lvert
    \begin{array}{ccc}
      x & \bar x & 1 \\
      y & \bar y & 1 \\
      z & \bar z & 1
    \end{array}
  \right\rvert. \]
\end{lemma}
This formula is often easier to apply if we shift $z$ to the point $0$ first, then shift back afterwards.

\section{Examples}
\begin{example}
  [MOP 2006]
  Let $H$ be the orthocenter of triangle $ABC$.
  Let $D$, $E$, $F$ lie on the circumcircle of $ABC$ such that $\ol{AD} \parallel \ol{BE} \parallel \ol{CF}$.
  Let $S$, $T$, $U$ respectively denote the reflections of $D$, $E$, $F$ across $\ol{BC}$, $\ol{CA}$, $\ol{AB}$.
  Prove that points $S$, $T$, $U$, $H$ are concyclic.
\end{example}
\begin{proof}
  Let $(ABC)$ be the unit circle and $h = a+b+c$.
  WLOG, $\ol{AD}$, $\ol{BE}$,  $\ol{CF}$ are perpendicular to the real axis (rotate appropriately); thus $d = \ol a$ and so on.
  Thus $s = b+c - bc \ol d = b+c-abc$ and so on; we now have
  \[ \frac{s-t}{s-u}=\frac{b-a}{c-a}\quad\text{and}\quad\frac{h-t}{h-u}=\frac{b+abc}{c+abc}. \]
  Compute \[ \frac{s-t}{s-u} : \frac{h-t}{h-u} = \frac{(b-a)(c+abc)}{(c-a)(b+abc)}
    = \frac{\left(\frac{1}{b}-\frac{1}{a}\right)\left(\frac{1}{c}+\frac{1}{abc}\right)}{\left(\frac{1}{c}-\frac{1}{a}\right)\left(\frac{1}{b}+\frac{1}{abc}\right)}
    \implies \frac{s-t}{s-u}:\frac{h-t}{h-u} \in \RR \]
  as desired.
\end{proof}


\begin{example}
  [Taiwan TST 2014]
  In $\triangle ABC$ with incenter $I$,
  the incircle is tangent to $\ol{CA}$, $\ol{AB}$ at $E$, $F$.
  The reflections of $E$, $F$ across $I$ are $G$, $H$.
  Let $Q$ be the intersection of $\ol{GH}$ and $\ol{BC}$, and let $M$ be the midpoint of $\ol{BC}$.
  Prove that $\ol{IQ}$ and $\ol{IM}$ are perpendicular.
\end{example}
\begin{figure}[ht]
  \centering
  \begin{asy}
    pair A = dir(110), B = dir(210), C = dir(330);
    pair I = incenter(A,B,C);
    pair E = foot(I,C,A);
    pair F = foot(I,A,B);
    pair D = foot(I,B,C);
    pair G = 2*I-E;
    pair H = 2*I-F;
    pair Q = extension(B,C,G,H);
    pair M = midpoint(B--C);
    draw(A--B--C--cycle);
    draw(H--Q);
    draw(incircle(A,B,C));
    draw(B--Q--I--M);
    dot("$A$", A, A);
    dot("$B$", B, dir(-90));
    dot("$C$", C, dir(-90));
    dot("$I$", I, dir(90));
    dot("$E$", E, dir(C+A-I));
    dot("$F$", F, dir(A+B-I));
    dot("$D$", D, dir(B+C-I));
    dot("$G$", G, dir(I-E));
    dot("$H$", H, dir(I-F));
    dot("$Q$", Q, dir(-90));
    dot("$M$", M, dir(-90));
  \end{asy}
\end{figure}
\begin{soln}
  Let $D$ be the foot from $I$ to $\ol{BC}$, and set $(DEF)$ as the unit circle.
  (This lets us exploit the results of \Cref{sec:unitcircle}.)
  Thus $\left\lvert d \right\rvert = \left\lvert e \right\rvert = \left\lvert f \right\rvert = 1$,
  and moreover $g = -e$, $h = -f$.
  Let $x = \ol d = \frac 1d$ and define $y$, $z$ similarly.
  Then
  \[ b = \frac{2}{\ol d + \ol f} = \frac{2}{x+z}. \]
  Similarly, $c = \frac{2}{x+y}$, so
  \[ m = \half(b+c) = \frac{1}{x+y} + \frac{1}{x+z} = \frac{2x+y+z}{(x+y)(x+z)}. \]
  Next, we have $Q = DD \cap GH$, which implies
  \[ q = \frac{dd(g+h)-gh(d+d)}{d^2-gh}
    = \frac{\frac{1}{x^2} \left( -\frac1y-\frac1z \right) - \frac{1}{yz} \frac{2}{x}}{\frac{1}{x^2} - \frac{1}{yz}}
    = \frac{2x+y+z}{x^2-yz}. \]
  so
  \[ m/q = \frac{x^2-yz}{(x+y)(x+z)}. \]
  Now,
  \[ \ol{m/q} = \frac{\frac{1}{x^2}-\frac{1}{yz}}{\left( \frac1x+\frac1y \right)\left( \frac1x+\frac1z \right)} = \frac{yz-x^2}{(x+y)(x+z)} = -m/q \]
  thus $m/q \in i \RR$, as desired.
\end{soln}

\begin{example}
  [USAMO 2012]
  Let $P$ be a point in the plane of $\triangle ABC$, and $\gamma$ a line through $P$.
  Let $A', B', C'$ be the points where the reflections of lines $PA, PB, PC$ with
  respect to $\gamma$ intersect lines $BC, AC, AB$ respectively.
  Prove that $A', B', C'$ are collinear.
\end{example}

\begin{figure}[ht]
  \centering
  \begin{asy}
    size(4cm);
    pair A = dir(110), B = dir(210), C = dir(330);
    dot("$A$", A, A);
    dot("$B$", B, dir(-90));
    dot("$C$", C, dir(-90));
    draw(A--B--C--cycle);
    pair P = 0.1*dir(40);
    dot("$P$", P, dir(-35));
    pair X1 = P + 1.5 * dir(10);
    pair X2 = P - 1.5 * dir(10);
    draw(X1--X2, dashed, Arrows);
    pair T = 2*foot(A,X1,X2)-A;
    pair A1 = extension(P,T,B,C);
    dot("$A'$", A1, dir(-40));
    draw(A--P--A1);
  \end{asy}
\end{figure}

\begin{soln}
  Let $p=0$ and set $\gamma$ as the real line.
  Then $A'$ is the intersection of $bc$ and $p\bar a$.
  So, using \Cref{prop:complex_intersect} we get
  \[ a' = \frac{\bar a(\bar b c - b \bar c)}{(\bar b - \bar c)\bar a-(b-c) a}. \]
  Note that
  \[ \bar a' = \frac{a (b \bar c - \bar b c)}{(b-c) a - (\bar b - \bar c)\bar a}. \]
  Thus by \Cref{thm:complex_shoelace}, it suffices to prove
  \[ 0 = \left\lvert
  \begin{array}{ccc}
    \frac{\bar a(\bar b c-b \bar c)}{(\bar b - \bar c)\bar a - (b-c) a} & \frac{a (b \bar c - \bar b c)}{(b-c) a - (\bar b - \bar c) \bar a} & 1 \\
    \frac{\bar b(\bar c a-c \bar a)}{(\bar c - \bar a)\bar b - (c-a) b} & \frac{b (c \bar a - \bar c a)}{(c-a) b - (\bar c - \bar a) \bar b} & 1 \\
    \frac{\bar c(\bar a b-a \bar b)}{(\bar a - \bar b)\bar c - (a-b) c} & \frac{c (a \bar b - \bar a b)}{(a-b) c - (\bar a - \bar b)\bar c} & 1
  \end{array}
  \right\rvert.
  \]
  This is equivalent to
  \[ 0 = \left\lvert
  \begin{array}{ccc}
    \bar a(\bar b c -b \bar c) & a (\bar b c - b \bar c) & (\bar b - \bar c) \bar a - (b-c) a \\
    \bar b(\bar c a -c \bar a) & b (\bar c a - c \bar a) & (\bar c - \bar a) \bar b - (c-a) b \\
    \bar c(\bar a b -a \bar b) & c (\bar a b - a \bar b) & (\bar a - \bar b) \bar c - (a-b) c \\
  \end{array}
  \right\rvert. \]
  Evaluating the determinant gives
  \[
  \sum_{\text{cyc}} ((\bar b - \bar c) \bar a - (b-c) a) \cdot - \left\lvert
    \begin{array}{cc}
      b & \bar b \\
      c & \bar c
    \end{array}
    \right\rvert \cdot \left( \bar c a - c \bar a \right)\left( \bar a b - a \bar b \right)
  \] or, noting the determinant is $b\bar c - \bar b c$ and factoring it out,
  \[
    (\bar b c - c \bar b)(\bar c a - c \bar a)(\bar a b - a \bar b) \sum_{\text{cyc}} \left( ab - ac + \bar c \bar a - \bar b \bar a \right) = 0.
  \qedhere \]
\end{soln}

\begin{example}
  [Taiwan TST Quiz 2014]
  Let $I$ and $O$ be the incenter and circumcenter of $ABC$.
  A line $\ell$ is drawn parallel to $\ol{BC}$ and tangent to the incircle of $ABC$.
  Let $X$, $Y$ be on $\ell$ so that $I$, $O$, $X$ are collinear and $\angle XIY = 90\dg$.
  Show that $A$, $X$, $O$, $Y$ are concyclic.
\end{example}
\begin{figure}[ht]
  \centering
  \begin{asy}
    pair A = dir(110), B = dir(220), C = dir(320);
    dot("$A$", A, A);
    dot("$B$", B, B);
    dot("$C$", C, C);
    draw(A--B--C--cycle);
    pair O = (0,0);
    dot("$O$", O, dir(-90));
    draw(unitcircle);
    draw(incircle(A,B,C), dotted);
    pair I = incenter(A,B,C);
    dot("$I$", I, dir(-90));
    pair B1 = 2*I-B, C1 = 2*I-C;
    pair X = extension(B1,C1,O,I);
    dot("$X$", X, dir(20));
    pair Y = 2*foot(circumcenter(A,X,O),X,2*I-foot(I,B,C))-X;
    dot("$Y$", Y, dir(190));
    draw(circumcircle(A,X,Y));
    draw(X--Y--I--cycle);
    pair P = IP(Line(I,Y),unitcircle);
    pair Q = 2*foot(O,I,P)-P;
    dot("$P$", P, dir(P-Q));
    dot("$Q$", Q, dir(Q-P));
    draw(P--Y,dashed); draw(I--Q,dashed);
    dot("$Y'$", IP(I--Q,B--C), dir(-110));
    pair X1 = extension(I,O,B,C);
    dot("$X'$", X1, dir(-75));
    draw(X1--I, dashed);
  \end{asy}
\end{figure}

\begin{soln}
  Let $X'$ and $Y'$ respectively denote the reflections of $X$ and $Y$ across $I$.
  Note that $X$, $Y$ lie on $\ol{BC}$.
  Also, let $P$, $Q$ be the intersections of $\ol{IY}$ with the circumcircle.

  Of course, $(ABC)$ is the unit circle.
  Let $j$ be the complex number corresponding to $I$ (to avoid confusion with $i = \sqrt{-1}$).
  Thus,
  \[ x' = \frac{\left( \ol b c - b \ol c \right)(j-0) - \left( \ol j 0 - j \ol 0 \right)(b-c)}{(\ol b - \ol c)(j-0) - (b-c) (\ol j - \ol 0)}
    = \frac{j \cdot \frac{c^2-b^2}{bc}}{j \cdot \frac{c-b}{bc} - (b-c) \ol j}
    = \frac{j(b+c)}{j + bc \ol j} . \]
  We seek $y'$ now. Consider the quadratic equation in $z$ given by
  \[ \frac{z-j}{j} + \frac{\frac1z - \ol j}{\ol j} = 0
    \iff z^2 - 2j z + j / \ol j = 0.\]
  Its zeros in $z$ are $p$ and $q$,
  which implies that $p+q = 2j$ and $pq = j / \ol j$
  (by Vieta!).
  From this we can compute
  \[ y' = \frac{pq(b+c)-bc(p+q)}{pq-bc}
    = \frac{j(b+c)-2bcj \ol j}{j - bc \ol j} = \frac{j(b+c)-2bcj \ol j}{j-bc\ol j}. \]
  which gives
  \[
    x = 2j - x' = \frac{j(2j-b-c+2bc \ol j)}{j + bc \ol j}
    \quad\text{and}\quad
    y = 2j - y' = \frac{j(2j-b-c)}{j - bc \ol j}.
  \]
  From this we can obtain
  \begin{align*}
    y-x &= j \cdot \frac{(2j-b-c)(j+bc\ol j)-(2j-b-c+2bc\ol j)(j-bc \ol j)}{(j-bc \ol j)(j+bc \ol j)} \\
    &= j \cdot \frac{2bc \ol j(2j-b-c)-2bc \ol j(j  - bc \ol j)}{(j-bc\ol j)(j+bc\ol j)} \\
    &= j \cdot \frac{2bc \ol j \left( j-b-c+bc \ol j \right)}{(j-bc\ol j)(j+bc\ol j)} \\
    X = \frac{y-x}{x} &= \frac{2bc \ol j \left( j-b-c+bc \ol j \right)}{(j-bc\ol j)(2j-b-c+2bc\ol j)} \\
    A = \frac{y-a}{a} &= \frac{j(2j-b-c-a)+abc\ol j}{a(j-bc\ol j)}
  \end{align*}
  We need to prove $X/A = \ol{X/A}$.
  Now set $a=x^2$, $b=y^2$, $c=z^2$, $j=-(xy+yz+zx)$, $\ol j = -\frac{x+y+z}{xyz}$
  (this is a different $x$, $y$ than the points $X$ and $Y$.)
  So, the above rewrites as
  \begin{align*}
    X &= \frac{2\frac{yz}{x}(x+y+z)(\frac{yz}{x}(x+y+z)+y^2+z^2+xy+yz+zx)}{\left( -\frac{yz}{x}(x+y+z)+xy+yz+zx \right)\left( y^2+z^2+2(xy+yz+zx)+2\frac{yz}{x}(x+y+z) \right)} \\
    &= \frac{2yz(x+y+z)\left( 2xyz+\sum_{\text{sym}} x^2y \right)}{(y+z)(x^2-yz)\left( x(y+z)(2x+y+z)+2yz(x+y+z) \right)} \\
    &= \frac{2yz(x+y+z)(x+y)(x+z)}{(x^2-yz)\left( (x^2+yz)(y+z)+(xy+yz+zx)(x+y+z) \right)} \\
    \intertext{and}
    A &= \frac{(xy+yz+zx)(x+y+z)^2-xyz(x+y+z)}{x^2(-(xy+yz+zx)+\frac{yz}{x}(x+y+z))}
    = \frac{(x+y+z)(x+y)(y+z)(z+x)}{x(yz-x^2)(y+z)} \\
    \intertext{thus}
    X/A &= \frac{-2xyz}{(x^2+yz)(y+z)+(x+y+z)(xy+yz+zx)} \\
    &= \frac{-\frac{2}{xyz}}%
    {(\frac{1}{x^2}+\frac1{yz})(\frac1y+\frac1z)  +  (\frac1x+\frac1y+\frac1z)(\frac1{xy}+\frac1{yz}+\frac1{zx})} = \ol{X/A}. \qedhere
  \end{align*}
\end{soln}


\section{Practice Problems}
\begin{enumerate}
  \ii Let $ABCD$ be cyclic.
  Let $H_A$, $H_B$, $H_C$, $H_D$ denote the orthocenters of $BCD$, $CDA$, $DAB$, $ABC$.
  Show that $\ol{AH_A}$, $\ol{BH_B}$, $\ol{CH_C}$, $\ol{DH_D}$ are concurrent.
  \ii (China TST 2011)
  Let $\Gamma$ be the circumcircle of a triangle $ABC$.
  Assume $\ol{AA'}$, $\ol{BB'}$, $\ol{CC'}$ are diameters of $\Gamma$.
  Let $P$ be a point inside $ABC$ and let $D$, $E$, $F$ be the feet from $P$ to $\ol{BC}$, $\ol{CA}$, $\ol{AB}$.
  Let $X$ be the reflection of $A'$ across $D$; define $Y$ and $Z$ similarly.
  Prove that $\triangle XYZ \sim \triangle ABC$.
  \ii In circumscribed quadrilateral $ABCD$ with incircle $\omega$,
  Prove that the midpoint of $\ol{AC}$ and the midpoint of $\ol{BD}$ are collinear with the center of $\omega$.
  \ii (Simson Line) Let $ABC$ be a triangle and $P$ a point on its circumcircle.
  \begin{enumerate}[(a)]
    \ii Let $D$, $E$, $F$ be the feet from $P$ to $\ol{BC}$, $\ol{CA}$, $\ol{AB}$.
    Show that $D$, $E$, $F$ are collinear.
    \ii Moreover, prove that the line through these points bisects $\ol{PH}$,
    where $H$ is the orthocenter of $ABC$.
  \end{enumerate}
  \ii (PUMaC Finals) Let $\gamma$ and $I$ be the incircle and incenter of triangle $ABC$.
  Let $D$, $E$, $F$ be the tangency points of $\gamma$ to $\ol{BC}$, $\ol{CA}$, $\ol{AB}$
  and let $D'$ be the reflection of $D$ about $I$.
  Assume $EF$ intersects the tangents to $\gamma$ at $D$ and $D'$ at points $P$ and $Q$.
  Show that $\angle DAD' + \angle PIQ = 180\dg$.
  \ii (Schiffler Point) Let triangle $ABC$ have incenter $I$.
  Prove that the Euler lines of $\triangle AIB$, $\triangle BIC$, $\triangle CIA$, $\triangle ABC$ are concurrent.
  \ii (USA TST 2014) Let $ABCD$ be a cyclic quadrilateral and let
  $E$, $F$, $G$, $H$ be the midpoints of $\ol{AB}$, $\ol{BC}$, $\ol{CD}$, $\ol{DA}$.
  Call $W$, $X$, $Y$, $Z$ the orthocenters of $AHE$, $BEF$, $CFG$, $DGH$.
  Prove that $ABCD$ and $WXYZ$ have the same area.
  \ii (Iran 2004) Let $O$ be the circumcenter of $ABC$.
  A line $\ell$ through $O$ cuts $\ol{AB}$ and $\ol{AC}$ at points $X$ and $Y$.
  Let $M$ and $N$ be the midpoints of $\ol{BY}$, $\ol{CX}$. Show that $\angle MON = \angle BAC$.
  \ii (APMO 2010) Let $ABC$ be an acute triangle, where $AB > BC$ and $AC > BC$.
  Denote by $O$ and $H$ the circumcenter and orthocenter.
  The circumcircle of $AHC$ intersects $AB$ again at $M$;
  the circumcircle of $AHB$ intersects $AC$ again at $N$.
  Prove that the circumcenter of triangle $MNH$ lies on line $OH$.
  \ii (Iran 2013) Let $ABC$ be acute, and $M$ the midpoint of minor arc $\widehat{BC}$.
  Let $N$ be on the circumcircle of $ABC$ such that $\ol{AN} \perp \ol{BC}$,
  and let $K$, $L$ lie on $AB$, $AC$ so that $\ol{OK} \parallel \ol{MB}$, $\ol{OL} \parallel \ol{MC}$.
  (Here $O$ is the circumcenter of $ABC$). Prove that $NK = NL$.
  \ii (MOP 2006) Cyclic quadrilateral $ABCD$ has circumcenter $O$.
  Let $P$ be a point in the plane
  and let $O_1$, $O_2$, $O_3$, $O_4$ be the circumcenters of $PAB$, $PBC$, $PCD$, $PDA$.
  Show that the midpoints of
  $\ol{O_1 O_3}$, $\ol{O_2O_4}$, $\ol{OP}$ are concurrent.
\end{enumerate}


\end{document}
