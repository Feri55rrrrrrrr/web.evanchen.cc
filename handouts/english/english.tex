\documentclass[11pt]{scrartcl}
\usepackage[sexy]{evan}

\begin{document}
\title{Remarks on English}
\subtitle{a.k.a.\ Advice for writing proofs}
\date{6 March 2020}
\maketitle

\epigraph{Exposition, criticism, appreciation, is work for second-rate minds.}
{G.\ H.\ Hardy}

\section{Grading}
Your score on an olympiad problem is a nonnegative integer at most $7$.
The unspoken rubric reads something like the following:
\begin{center}
\begin{tabular}[h]{ll}
  & Description \\ \hline
  \boldmath{$7^\ast$} & Problem was solved \\
  $6$ & Tiny slip (and contestant could repair) \\
  $5$ & Small gap or mistake, but non-central \\ \hline
  $2$ & Lots of genuine progress \\
  \boldmath{$1^\ast$} & Significant non-trivial progress \\
  \boldmath{$0^\ast$} & ``Busy work'', special cases, lots of writing
\end{tabular}
\end{center}
The ``default'' scores are starred above.
Note that, unlike high school English class or the SAT essay,
you don't get points just because you wrote a lot!

In theory, your solutions to olympiads are graded solely based on math.
In practice, style still does play a role in some ways:
the harder your solution is to understand,
the less likely the grader is to understand you,
and the less likely you are to earn points you deserve.\footnote{In addition,
  poorly written solutions make the graders sad,
  and you wouldn't want that, would you?}

\section{Stylistic suggestions}
Here are some tips of mine that I don't think are stressed enough.

\subsection{Never write wrong math}
This is much more of a math issue than a style issue:
you can lose all of your points for making false claims.
Personally, I often \emph{stop reading} a solution if it makes
an egregiously false claim: if someone claims that some fixed
point is the incenter of $ABC$,
when it's actually the arc midpoint,
then I know the solution isn't going to have any substantial progress.

As a special case, don't say something that is partially true
and then say how to fix it later.
At best this will annoy the grader;
at worst they may get confused and think the solution is wrong.

\subsection{Emphasize the point where you cross the ocean}
Solutions to olympiad problems often involve a few key ideas,
with the rest of the solution being checking details.
You want graders to immediately see all the key ideas in the solution:
this way, they quickly have a high-level understanding of your approach.

Let me share a quote from Scott Aaronson:
\begin{quote}
  Suppose your friend in Boston blindfolded you,
  drove you around for twenty minutes,
  then took the blindfold off and claimed you were now in Beijing.
  Yes, you do see Chinese signs and pagoda roofs,
  and no, you can't immediately disprove him ---
  but based on your knowledge of both cars and geography,
  isn't it more likely you're just in Chinatown?
  \dots
  \textbf{We start in Boston, we end up in Beijing,
  and at no point is anything resembling an ocean ever crossed.}
\end{quote}
Olympiad solutions work the same way:
a geometry solution might require a student to do some angle chasing,
use Fact 5 to deduce that two triangles are congruent,
and then finish by doing a little more angle chasing.
In that case, you want to highlight the key step of proving the two triangles
were congruent, so the grader sees it immediately and can say
``okay, this student is using this approach''.

Ways that you can highlight this are:
\begin{itemize}
  \ii Isolating crucial steps and claims as their own
  \textbf{lemmas}.\footnote{This is often useful for another reason:
    breaking the proof into individual steps.
    The complexity of understanding a proof grows super-linearly
    in its length; therefore breaking it into smaller chunks
    is often a good thing.}
  \ii Using \textbf{claims} to say what you're doing.
  Rather than doing angle chasing and writing
  ``blah blah blah, therefore $\triangle M_B I_B M \sim \triangle M_C I_C M$'',
  consider instead ``We claim $\triangle M_B I_B M \sim \triangle M_C I_C M$, proof''.
  \ii \textbf{Displaying} important equations.
  For example, notice how the line
  \begin{equation}
    \triangle M_B I_B M \sim \triangle M_C I_C M
    \label{eq:displaytri}
  \end{equation}
  jumps out at the reader.
  You can even number such claims to reference them letter,
  e.g.\ ``by \eqref{eq:displaytri}''.
  This is especially useful in functional equations.
  \ii Just \textbf{say it}!
  Little hints like ``the crucial claim is $X$''
  or ``the main idea is $Y$'' are immensely helpful.
  Don't make $X$ and $Y$ look like another intermediate step.
\end{itemize}

\subsection{``Find all\dots''}
Many problems will ask you to ``find all objects satisfying some condition''
(for example, functional equations, Diophantine equations).
For any solution of this form, I strongly recommend that you
structure your solution as follows:
\begin{itemize}
  \ii Start by writing ``\textbf{We claim the answer is \dots}''.
  \ii Then, say ``\textbf{We prove these satisfy the conditions}'', and do so.
  For example, in a functional equation with answer $f(x) = x^2$,
  you should plug this $f$ back in and verify the equation is satisfied.
  Even if this verification is trivial,
  you must still explicitly include it,
  because it is part of the problem.
  \ii Finally, say ``\textbf{Now we prove these are the only ones}''
  and do so.
\end{itemize}

Similarly, some problems will ask you to
``find the minimum/maximum value of $X$''.
In such situations, I strongly recommend you write your solution as follows:
\begin{itemize}
  \ii Start by writing ``\textbf{We claim the minimum/maximum is \dots}''.
  \ii Then, say ``\textbf{We prove that this is attainable}'',
  and give the construction (or otherwise prove existence).
  Even if this verification is trivial,
  you must still explicitly include it,
  because it is part of the problem.
  \ii Finally, say ``\textbf{We prove this is a lower/upper bound}'', and do so.
\end{itemize}

Failing to do one of the steps mentioned above is a classic newbie mistake.
Make it abundantly clear to the grader that you know the difference
between a bound and a maximum.

\subsection{Leave space}
Most people don't leave enough space.
This makes solutions hard to read.

Examples of things you can do:
\begin{itemize}
  \ii Skip a line after paragraphs.
  Use paragraph breaks more often than you already do.

  \ii If you isolate a specific \textbf{lemma or claim} in your proof,
  then it should be on its own line,
  with some whitespace before and after it.

  \ii Any time you do \textbf{casework},
  you should always split cases into separate paragraphs
  or bullet points.
  Make it visually clear when each case begins and ends.

  \ii Display important equations,
  rather than squeezing them into paragraphs.
  If you have a long calculation,
  then do an aligned display\footnote{This is the
    \texttt{align*} environment,
    for those of you that like \LaTeX.}
  rather than squeezing it into a paragraph.
  For example, instead of writing $0 \le {(a - b)^2} = {(a + b)^2} - 4ab = {(10 - c)^2} - 4\left( {25 - c(a + b)} \right) = {(10 - c)^2} - 4\left( {25 - c(10 - c)} \right) = c(20 - 3c)$, write instead
  \begin{align*}
    0 &\le {(a - b)^2}\\
     &= {(a + b)^2} - 4ab\\
     &= {(10 - c)^2} - 4\left( {25 - c(a + b)} \right)\\
     &= {(10 - c)^2} - 4\left( {25 - c(10 - c)} \right)\\
     &= c(20 - 3c).
  \end{align*}
\end{itemize}

\subsection{Other things}
Try to have nice handwriting.
Include a large, scaled diagram in geometry problems\footnote{And
try to not have circles which look like potatoes.}.
Leave 1-inch (or more) margins.
% Try to avoid making typos.
Write your proofs forwards even if you solved the problem backwards.
If you need to cite a theorem, say clearly how you're doing so.
Use variable names at your discretion.
Strike out and cross out unwanted parts of your solution (don't scribble).

I'm sure someone has told you these before.
If not, consider reading
\url{https://www.artofproblemsolving.com/articles/how-to-write-solution}.

\section{Example}
Consider the following problem.
\begin{mdframed}
  \textbf{(USAMO 2014)}
  Let $a$, $b$, $c$, $d$ be real numbers such that $b-d \ge 5$
  and all zeros $x_1$, $x_2$, $x_3$, and $x_4$ of the
  polynomial $P(x)=x^4+ax^3+bx^2+cx+d$ are real.
  Find the smallest value the product
  \[ (x_1^2+1)(x_2^2+1)(x_3^2+1)(x_4^2+1) \]
  can take.
\end{mdframed}

Here are two ways you could write the solution.\footnote{Former
solution worsened June 2018, with suggestions from Mitchell Lee.}
\begin{mdframed}
  \textbf{Pretty poor solution}. \; \,
  $x_j^2+1 = (x-i)(x+i) \forall j$
  $\implies \prod x_j^2+1 = \prod (x_j+i)(x_j-i) = P(i)P(-i)$
  so $(b-d-1)^2 + (a-c)^2$.
  $\because x_j = 1 \rightarrow 16$ and $\binom42-1 = 5$.
  $b-d \ge 5$, so $\ge 16$.
\end{mdframed}

\begin{mdframed}
  \textbf{Better solution}.
  The answer is $\boxed{16}$.
  This can be achieved by taking $x_1 = x_2 = x_3 = x_4 = 1$,
  whence the product is $2^4 = 16$, and $b-d = \binom42-1 = 5$.

  Now, we prove this is a lower bound.
  The key observation is that
  \[ \prod_{j=1}^4 \left( x_j^2 + 1 \right)
    = \prod_{j=1}^4 (x_j - i)(x_j + i)
    = P(i)P(-i) = |P(i)|^2. \]
  Consequently, we have
  \begin{align*}
    \left( x_1^2 + 1 \right)
    \left( x_2^2 + 1 \right)
    \left( x_3^2 + 1 \right)
    \left( x_4^2 + 1 \right)
    &= (b-d-1)^2 + (a-c)^2 \\
    &\ge (5-1)^2 + 0^2 = 16.
  \end{align*}
  This proves the lower bound.
\end{mdframed}

These solutions have the same mathematical content.
But notice how in the better solution:
\begin{itemize}
  \ii The second solution makes it clear
  from the beginning what the answer is, and what the equality case is.
  (The first solution mixes these together.)
  \ii Moreover, the main idea (of factoring with $i$) is explicitly labeled,
  so that even if you have never seen the problem before,
  you can tell at a glance what the main idea of the solution is.
  \ii The equations are displayed in the second solution,
  making them much easier to read than in the first.
\end{itemize}
The second solution, despite being twice as ``long'',
is by far faster to read than the first solution.
In this case, the difference is not so bad because the
problem and solution are quite short.
However, in more involved problems the ``not-so-good solution''
becomes the ``completely unreadable solution''.

\section{Funny quote}
From \href{https://www.gleech.org/better-maths}{Steven Witten}:
\begin{quote}
  \slshape
  Imagine I asked you to learn a programming language where:
  \begin{itemize}
  \ii All the variable names were a single letter,
  and where programmers enjoyed using foreign alphabets,
  glyph variation and fonts to disambiguate their code from meaningless gibberish.
  \ii None of the functions were documented,
  and instead the API docs consisted of circular references to other pieces of similar code,
  often with the same names overloaded into multiple meanings, often impossible to Google.
  \ii None of the sample code could be run on a typical computer;
  in fact, most of it was pseudo-code lacking a definition of input and output,
  or even the environment it was supposed to run.
  \end{itemize}
\end{quote}

\appendix

\section{Notes specific to mathematical competitions}
Up until now I've given my advice for how to write solutions well.
But I know a lot of you are specifically interested in olympiad grading,
so here are a few quick remarks to that end.
These comments are meant for USA(J)MO in particular
but should apply to other respectable contests as well.

\subsection{More examples of decent write-ups}
I should note that on my website
\begin{center}
  \url{https://web.evanchen.cc/problems.html}
\end{center}
there are a very large number of solutions written by me
to past problems on the USAMO, IMO, USA TST(ST), etc.
In particular, all USAMO and IMO problems since the year 2000
are present.

Not all the solutions are complete
(some of them are just outlines),
but I think the majority of them are full write-ups,
and these can help provide more examples of solutions
that you can compare to or model your own work after.

\subsection{How much detail to include}
A common question I get is what the minimum amount of detail needed
to get full marks for a solution is.
The answer is simple: enough to convince the grader you solved the problem.

There is a myth that, sort of like your high school English or math teacher,
you can lose points for ``not writing enough'' or not having certain key words
or leaving out details that were obvious to everyone.
This is not really how it works.
USAMO graders are interested in whether you solved the problem
rather than your ability to fill pages with ink.

Basically, \alert{you lose points if a student who did NOT solve the problem
could have written the same words as you}.
For example, whenever you say something like ``it's easy to see $X$'',
the grader has to ask whether you actually understand why $X$ is true,
or don't know and are just bluffing.
So that's always the criteria you should have in your head
when deciding what needs to be written out in full.

As a very loose rule of thumb,
the official solutions file for USAMO (published by MAA)
is about as terse as you can be.

\subsection{Citing lemmas}
In general it is usually okay to cite a result that is
(i) named, and (ii) does not trivialize the given problem.
Anything outside this scope is a ``grey area''
and I don't want to commit to a hard set of guidelines.

However, the main thing I want to say is that
\alert{if in doubt, outline a proof}.
You don't have to choose between the extremes
``say absolute zero'' and ``prove quoted lemma in full gory detail''.
It's better to just include a couple lines giving the overall idea of the proof
to show that you \emph{could} write it out if you wanted to,
but are omitting it because the result is already known.

\subsection{Fake-solving problems}
With all that said, I would say in the end,
when \alert{people don't get the points they expect,
it's because their solution is actually wrong or incomplete},
not because they wrote it poorly.
This is true something like 90\% of the time, maybe more.

Some common ways to lose most or all of your points
by virtue of not having solved the problem:
\begin{itemize}
  \ii Flipping an inequality sign.
  \ii Not understanding what the word ``function'' means
  in a functional equation.
  \ii Making some assumption that seems intuitive,
  but actually requires justification (and is the main difficulty of the problem).
  \ii Stating key assertions with no proof
  (often which are equivalent to the problem).
  \ii Making some actual logical error
  (for example, the so-called ``pointwise trap'').
  \ii Missing some case or possibility that the student didn't realize existed.
  \ii Not understanding the problem statement altogether
  (for example, not knowing that ``find all'' problems have two parts,
  and only doing one direction).
\end{itemize}
Some examples of USAMO problems that are notorious for generating wrong solutions:
USAMO 2003/6, USAMO 2007/2, USAMO 2010/3, USAMO 2016/4.

I should say there is no shame in having an incorrect solution to a problem,
it really happens to everyone more often than anyone wants to admit.
Just don't delude yourself into thinking that you lost points
you deserved because the graders didn't like your style.

\section{Checking your work}
\subsection{How to check your solutions during the year}
When you are practicing during the year,
the best way to get feedback on proofs is to have a friend/coach
who can check your work and provide suggestions.
But the supply of people willing to do this is admittedly very low,
so most people are not so lucky to have access to feedback.

If you don't have access to such feedback,
I suggest the following second-best measures.
\begin{itemize}
  \ii Write up neatly.
  The more clear your write-up is,
  the more likely you are to catch your own mistakes.

  \ii Write up your solutions to past IMO/USAMO problems in full,
  and post them on the Art of Problem Solving forum
  under the thread for that problem (not the wiki).
  By Cunningham's Law, if you have a blatantly wrong solution,
  someone will often point it out within a few hours.

  \ii Compare your solutions to others posted.
  Often, a problem will have essentially only a few approaches,
  and you'll find another user who had more or less
  the same approach.\footnote{There are unfortunately some problems,
    like USAMO 2017/1,
    where so many different solutions are possible
    that any two people are likely to have different approaches.}
  This serves as a sanity check that what you have does work.

  If you find your solution is way shorter or simpler
  than everyone else, then you have good reason to be suspicious.
  Look for the ocean-crossing point in other people's solutions.
  Why did they have to work so hard there, while you did not?
  Often, that's where the mistake will be.
\end{itemize}

\subsection{How to check your solutions during a contest}
Of course, it is critical to eventually
be able to check your own work independently
without consulting other people.
The IMO does not have live feedback;
by the time someone tells you about a mistake, it is too late!

If you are a beginner it might take a while to reach this stage,
but you should set this as a goal for where you want to end up.
It is easier than you might expect ---
as you naturally get better at solving problems, your instincts
about the correctness of proofs will automatically develop too.

During the contest, the only advice I have is
``write clearly and carefully''
(which is why developing these habits pays off later).
I cannot tell you how many times I realized only during the write-up phase
that the ``solution'' I thought I had was actually flawed.

\end{document}

Didn't mention: scratch paper, style scores
